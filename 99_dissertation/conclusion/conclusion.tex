\chapter{Conclusion}

\textit{
    ``ESTRAGON: (looking at the tree). What is it? \\
    VLADIMIR: It's the tree. \\
    ESTRAGON: Yes, but what kind? \\
    VLADIMIR: I don't know. A willow. \\
    Estragon draws Vladimir towards the tree. They stand 
    motionless before it. Silence. \\
    \textemdash\ ``Waiting for Godot'' by Samuel Beckett
} \citep{beck1982waiting}

This research use Amartya's entitlement approach to focus on famine, and then the population change during the 19th century in Ireland. Unlike typical famine and population research, which doing the research under a FAD theory framework, this research rebut it first. During quantitative progress, these arguments are drew: 

Firstly, FAD is not suitable for Irish case although potato blight happened. Using grain import and export amount as control variables, and grain planting acreage as key independent variable, this paper conduct a LM regression and shows a non-significant coefficient in grain acreage, which shows there is no relationship between grain acreage and population change in the 19th century Ireland.

Secondly, entitlement approach, both on the effect of famine and of the population development, is an important mechanism to explain. With the same control variables, and grain price, ground rent, tax status, wage and poor law as key independent variables, this paper conduct a GAM regression and shows a significant coefficient in potato price, wage and poor law, which prove the trade-based entitlement, own-labour entitlement and transfer and inheritance entitlement are making influence in the population change during the 19th century Ireland respectfully.
