\chapter{Data}

\vspace{0pt}

% 使用 longtable 和 booktabs 创建三线表
\begin{longtable}{ccc}
    \caption{Data and Sources} \\
    \toprule % 表格顶部线
    \textbf{Data} & \textbf{Time} & \textbf{Sources} \\
    \midrule % 表格标题下方线
    \endfirsthead

    \caption[]{(Continued)} \\
    \toprule
    \textbf{Data} & \textbf{Time} & \textbf{Sources} \\
    \midrule
    \endhead

    \midrule
    \multicolumn{3}{r}{\textit{Continued on next page}} \\
    \midrule
    \endfoot

    \bottomrule % 表格底部线
    \endlastfoot

    Row 1 Col 1 & Row 1 Col 2 & Row 1 Col 3 \\
    Row 2 Col 1 & Row 2 Col 2 & Row 2 Col 3 \\
    Row 3 Col 1 & Row 3 Col 2 & Row 3 Col 3 \\
    Row 4 Col 1 & Row 4 Col 2 & Row 4 Col 3 \\
    Row 5 Col 1 & Row 5 Col 2 & Row 5 Col 3 \\
    % 添加更多行以确保表格跨多页
    Row 6 Col 1 & Row 6 Col 2 & Row 6 Col 3 \\
    Row 7 Col 1 & Row 7 Col 2 & Row 7 Col 3 \\
    Row 8 Col 1 & Row 8 Col 2 & Row 8 Col 3 \\
    Row 9 Col 1 & Row 9 Col 2 & Row 9 Col 3 \\
    Row 10 Col 1 & Row 10 Col 2 & Row 10 Col 3 \\
    Row 11 Col 1 & Row 11 Col 2 & Row 11 Col 3 \\
    Row 12 Col 1 & Row 12 Col 2 & Row 12 Col 3 \\
    Row 13 Col 1 & Row 13 Col 2 & Row 13 Col 3 \\
    Row 14 Col 1 & Row 14 Col 2 & Row 14 Col 3 \\
    Row 15 Col 1 & Row 15 Col 2 & Row 15 Col 3 \\
    Row 16 Col 1 & Row 16 Col 2 & Row 16 Col 3 \\
    Row 17 Col 1 & Row 17 Col 2 & Row 17 Col 3 \\
    Row 18 Col 1 & Row 18 Col 2 & Row 18 Col 3 \\
    Row 19 Col 1 & Row 19 Col 2 & Row 19 Col 3 \\
    Row 20 Col 1 & Row 20 Col 2 & Row 20 Col 3 \\
    Row 21 Col 1 & Row 21 Col 2 & Row 21 Col 3 \\
    Row 22 Col 1 & Row 22 Col 2 & Row 22 Col 3 \\
    Row 23 Col 1 & Row 23 Col 2 & Row 23 Col 3 \\
    Row 24 Col 1 & Row 24 Col 2 & Row 24 Col 3 \\
    Row 25 Col 1 & Row 25 Col 2 & Row 25 Col 3 \\
    Row 26 Col 1 & Row 26 Col 2 & Row 26 Col 3 \\
    Row 27 Col 1 & Row 27 Col 2 & Row 27 Col 3 \\
    Row 28 Col 1 & Row 28 Col 2 & Row 28 Col 3 \\
    Row 29 Col 1 & Row 29 Col 2 & Row 29 Col 3 \\
    Row 30 Col 1 & Row 30 Col 2 & Row 30 Col 3 \\
\end{longtable}

% \begin{table}[h]
%     \centering
% 	\begin{threeparttable}
%     \caption{Data and Sources}
% 		\begin{tabular}{lp{5cm}p{5cm}}
% 			\toprule
% 			Data & Time & Sources \\
% 			\midrule
% 			Population & 1821 \textendash\ 1900 & Ireland census\tnote{a} and the estimated population\tnote{b}\\
% 			& & \\
% 			Wage & 1821 \textendash\ 1900 & \citep{d1989wages} \& \citep{bishop1915history}\\
% 			& & \\
% 			Ground Rent & Key grouping, like landlord class or the UK government & \citep{smith2005reckoning} \\
% 			& & \\
% 			Grain Price & Oat Price &  \\
%             & Barley Price & \\
%             & Potato Price & \\
%             & Wheat Price & \\
% 			& & \\
%             Grain Plant Acre & Oat Acre & \\
%             & Barley Acre & \\
%             & Potato Acre & \\
%             & Wheat Acre & \\
%             & & \\
%             Grain Import & Oat Import & \\
%             & Barley Import & \\
%             & Potato Import & \\
%             & Wheat Import & \\
%             & & \\
%             Grain Export & Oat Export & \\
%             & Barley Export & \\
%             & Potato Export & \\
%             & Wheat Export & \\
%             & & \\
% 			1980s: post-revisionist & Emotional description also blame UK government & \citep{hamera2011outline} \\
% 			& & \\
% 			1980s: diverse & Malthus population theory & \citep{o2009food} \& \citep{mcgregor1989demographic} \& \citep{weir1991malthus} \\
% 			& &  \\
% 			& Anti-Malthus theory & \citep{o1983malthus} \& \citep{mokyr1980malthusian} \& \citep{guinnane1994great}\\
% 			& & \\
% 			& Blight biological analysis & \citep{donnelly2011irish}\\
% 			& & \\
% 			& Foucault's bio-politics and colonial perspective & \citep{nally2008coming} \& \citep{kennedy2020beckett} \& \citep{madden2016aids} \\
% 			\bottomrule
% 		\end{tabular}
% 		\begin{tablenotes}
% 			\item[a] The original newspaper mentioned: \textit{Free importation of corn into this union is essentially necessary \textendash\ [\ldots] any attempt to re-impose a duty on the importation of food can only [\ldots]  tend to the starving of the people. Poor law [\ldots] relieves the struggling farmer of a heavy burden he had hitherto.} \citep{1850_01_05_news}
			
% 			\vspace{7pt}

% 			\item[b] The famine literature few. The quantity and quality of work on the famine sparse: \textit{The two standard books of the Great Famine, [\ldots] the chapters were uneven in quality and lacked coherence (some lacked footnotes, some were lost).} \citep{kinealy2017great}

% 		\end{tablenotes}
% 	\end{threeparttable}
% \end{table}

\vspace{.3cm}





