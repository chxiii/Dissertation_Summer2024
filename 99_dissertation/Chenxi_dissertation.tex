\documentclass[a4paper,oneside,12pt]{book}

%----------------------------------------------------------------------------------------
%	README!
%   Welcome. It's worth having a read through this file
%   to set up the broad parameters, such as the name of
%   the degree, the school/department, the type of work
%   (dissertation/Final Year Project/report, etc. as well
%   as your own details.
%----------------------------------------------------------------------------------------

%----------------------------------------------------------------------------------------
%	COVER PAGE
%   The cover page is laid out in title/title.tex. You can choose a colour
%   or black and white logo
%----------------------------------------------------------------------------------------

%----------------------------------------------------------------------------------------
%	THESIS INFORMATION
%   Put title, author name, degree, type of work, school, department in here
%   It will be used for the title page and for the embedded PDF information
%----------------------------------------------------------------------------------------

\newcommand{\thesistitle}{Entitlement Approach \textendash\ \\An Example of the Irish Famine and Population Change, 
1821 \textendash\ 1900} % Your thesis title, this is used in the title and abstract
\newcommand{\degree}{M.Sc. Applied Social Data Sciences} % Your degree name, this is used in the title page and abstract
\newcommand{\typeofthesis}{Dissertation} % dissertation, Final Year Project, report, etc.
\newcommand{\authorname}{Chenxi Li} % Your name, this is used in the title page and PDF stuff
%% Do not put your Student ID in the document, as TCD will not publish
%% documents that contain both your name and your Student ID.
\newcommand{\authorid}{23330541}
\newcommand{\keywords}{Entitlement Approach, } % Keywords for your thesis
\newcommand{\school}{\href{https://www.tcd.ie/Political_Science/}{School of Political}} % Your school's name and URL, this is used in the title page

%% Comment out the next line if you don't want a department to appear
%\newcommand{\department}{\href{http://researchgroup.university.com}{Prof. Peter Dunne}} % Your research group's name and URL, this is used in the title page

\AtBeginDocument{

\hypersetup{pdftitle=\thesistitle} % Set the PDF's title to your title
\hypersetup{pdfauthor=\authorname} % Set the PDF's author to your name
\hypersetup{pdfkeywords=\keywords} % Set the PDF's keywords to your keywords
\hypersetup{pdfsubject=\degree} % Set the PDF's keywords to your keywords
}

%% Language and font encodings
\usepackage[T1]{fontenc} 
\usepackage[utf8]{inputenc}
\usepackage[english]{babel}
\usepackage{ragged2e} %allows for text alignment preferences

%% Bibliographical stuff
\usepackage[round,sort,comma]{natbib}
\usepackage[hang, bottom]{footmisc} % no space in footnote

%furmula
\usepackage{amsmath}

%% Document size
% include show frame as an option if you want to see the boxes
\usepackage[a4paper, top=2.56cm,bottom=2.56cm,left=2.56cm,right=2.56cm, head = 16pt]{geometry}
\setlength{\marginparwidth}{2cm}
%% Useful packages

\usepackage[autostyle=true]{csquotes} % Required to generate language-dependent quotes in the bibliography
\usepackage[pdftex]{graphicx}
\usepackage[colorinlistoftodos]{todonotes}

\usepackage{pdflscape}

\usepackage{xcolor}
\usepackage{caption} % if no caption, no colon
%\usepackage{sfmath} %use sans-serif in the maths sections too
\usepackage[parfill]{parskip}    % Begin paragraphs with an empty line rather than an indent
\usepackage{setspace} % to permit one-and-a-half or double spacing
\usepackage{enumerate} % fancy enumerations like (i) (ii) or (a) (b) and suchlike
\usepackage{booktabs} % To thicken table lines
\usepackage{threeparttable}
\usepackage{fancyhdr}

% for table
\usepackage{longtable} % long table
\usepackage{threeparttablex}

% hyper to the reference
\usepackage[colorlinks=true, allcolors=black]{hyperref}

\pagestyle{fancy}
\fancyhf{} % sets both header and footer to nothing
\renewcommand{\headrulewidth}{0pt}
\cfoot{\thepage}


\usepackage{mathpazo} % Use the Palatino font by default if you prefer it to Computer Modern

\renewcommand{\theequation}{\arabic{equation}} %% use continuous equation numbers

%% Format Chapter headings appropriately
\usepackage{titlesec}
\definecolor{tcdblue}{cmyk}{0.94, 0.38, 0, 0.27}
\newcommand{\hsp}{\hspace{5pt}}

\titleformat{\chapter}[hang]
	{\centering\LARGE\bfseries\itshape}
	{Chapter \thechapter\hsp\textcolor{tcdblue}{|}\hsp}
	{0pt}{\LARGE\bfseries\itshape}

\titlespacing{\chapter}{0pt}{0pt}{10pt}


\title{\thesistitle}

\author{\authorname}

\frontmatter

\begin{document}

\input{title/title.tex}
\pagenumbering{roman}
\doublespacing

\newpage

\section*{Declaration}
I hereby declare that this \typeofthesis\ is entirely my own work and that it has not been submitted as an exercise for a degree at this or any other university.

I have read and I understand the plagiarism provisions in the General Regulations of the University Calendar for the current year, found at\\
 \url{http://www.tcd.ie/calendar}.

I have also completed the Online Tutorial on avoiding plagiarism Ready Steady Write' and located at \\
 \url{http://tcd-ie.libguides.com/plagiarism/ready-steady-write}.

\vspace{.3cm}
\rule{10cm}{.3pt}

\begin{flushleft}
	\begin{minipage}{\linewidth}
		\textbf{Signature:} 
		\raisebox{-0.3\height}{\includegraphics[width=0.15\linewidth]{signature.png}}
	\end{minipage}
\end{flushleft}
\textbf{Date: } \today
\vspace{.3cm}

\newpage

% \section*{Acknowledgements}
% I would like to thank my supervisor, \textbf{Prof. Martina Kirhberger}, for her guidance through each stage of this dissertation.

\chapter{Abstract}

Using Amartya's entitlement approach, this paper research on 19th century's Ireland population change trend and its mechanism. FAD theory predicts a casual relationship between food shortage and death, while a large number of empirical studies refute this argument and point out another fact identical with the entitlement approach, which is the casual relationship between the entitlements failure and the death. From Foucault's bio-politics to population and political economics, all point to an endogenous mechanism of demographic change related to entitlements and increase.

This paper seeks to answer these two research questions: (1) Could 19th century Ireland, based on the well known potato late blight famine, still be a refutation of FAD theory? (2) When using entitlement approach, what kind of entitlement failures led to sustained negative population growth in 19th century Ireland?

With historical data and GAM regression, (1) when controlling for the amount of grain imported and exported, the acreage of food and the population change do not show significance. (2) Potato price, wage and the Poor Law are significant in predicting population change, where there is a negative relationship between potato price and population growth, and positive relationship between wage and Poor Law.

The results of the study suggest that although Ireland suffered from late potato blight in the 19th century, it was still not food shortages that led to negative population growth, but rather entitlement failures; furthermore, the failure of trade-based entitlements was the most significant cause of negative population growth, i.e., the unreasonable price of potatoes. The increase of own-labour entitlements, i.e., the increase of the wage, and the transfer and inheritance entitlements, i.e., the poor law,  has slowed down the process of negative population growth.

\newpage
%\raggedright %\raggedright turns off justification and hypenation

\setlength{\parskip}{5pt}
%\setlength{\parsep}{0pt}

\newpage \tableofcontents
\newpage \listoffigures
\newpage \listoftables

\newpage

\vspace{2cm}

\mainmatter
\documentclass[a4paper,oneside,12pt]{book}

\begin{document}

As

\end{document}
\chapter{Literature Review}

\vspace{.3cm}

\textit{Famines imply starvation, but not vice versa, and starvation implies poverty, but not vice versa.\\
\textemdash\ Amartya Sen}

\section{Famine Outline}

Irish Lumper potato, 

\section{Refuting some hypotheses}

This part I will refute some hypothesis of famine origin. Many people regard single factor as the root of the Great Famine.

\subsection{Potato Blight}

In Nature journal,

1845 June Belgium, August France, August South of UK, September Ireland





1. Blame potato blight as the only origin of famine

People believe potato blight was responsible for the Irish Great Famine. 

lumper potato

Blight became a semi-permanent fixture until the end of the century, when effective treatments were found \citep{o1994economic}.

2. Ireland have the bad land quality.

\section{Entitlement Approach}

I will operationalize entitlement approach into these 4 dimensions according to the book:

(1) trade-based entitlement: price, grain amount, 

(2) production-based entitlement: tax policy

(3) own-labour entitlement: wage, land own amount, poor law

(4) inheritance and transfer entitlement: none, hard to get data




\chapter{Data}

\textit{
``Malone: Me father died of starvation in Ireland in the Black 47. Maybe you've heard of it.\\
Violet: The Famine?\\
Malone: No, the starvation. When a country is full of food, and exporting it, there can be no famine. Me father was starved dead; and I was starved out to America in me mother's arms''.\\
\textemdash\ ``Man and Superman'' by George Bernard Shaw
}
\vspace{.2cm}

The Irish famine should be examined in the context of the entire 19th century history, rather than discussing the years of the famine alone, which on the one hand would lead to an excessively small sample size and thus ineffective statistics modeling, and on the other hand would lead to a entitlement approach that cannot be analyzed in the context of both positive and negative scenarios of population growth and population decline. Therefore, in collecting data, this paper adopts the strategy of collecting data from 1821 to 1900, where 1821 is the year of the first Irish census with complete documentation, and 1900 marks the end of Ireland's troubled 19th century.

In addition, this paper used population change \textendash\ more specifically, the difference between current year's population and previous year's population \textendash\ as the dependent variable to measure the impact of the entitlement approach on population gap, whether it be pre-famine or post-famine sustained growth or decline, thus  realizing out the causal inference between entitlement approach and population change.

The dataset consists of 26 variables, including 23 dataset variables, with the continuous variable population, various cereal prices, various cereal acreage, various cereal imports and exports, land tax, wages and the categorical variable of if government taxed tithe, also 3 constructed variables, including cereal prices summed up except potatoes, cereal acreage summed up, and the difference in cereal imports and exports. Each year is an observation, totaling 80 observations from 1821 to 1900.

\section{Data Sources}
\vspace{0pt}
The data come from several primary sources, including (1) census data, (2) economic history research papers, and (3) original archival material from the National Library Ireland. Many materials only covered a few years, so this paper filled in the data by combining various materials. For example, regarding the price of oats, the data from 1821 to 1828 were obtained from Daniel's 2021 research, the data from 1829 to 1859 were obtained from Vamplew's 1980 research, and the data from 1850 to 1900 were obtained from Tuner's 1987 research. 

When splicing material from different sources, this paper performs cross validation between the data to ensure accuracy. For example, when both D'Arcy and Bisshop documented wage conditions in 19th century, this paper verified the consistency of the overlapping data from the two papers, and only spliced the data after ensuring. 

Below are all the variables and their sources: 

\vspace{7pt}

\begin{spacing}{1}
\begin{ThreePartTable}
    \begin{TableNotes}
        \begin{spacing}{1}
        \vspace{7pt}
        \item[a] \textit{Irish census through history can be found in \href{https://www.cso.ie/en/statistics/historicalreports/}{CSO}. In 1851 census, there is a chapter discussing the differences between 1841 and 1851 to show the influence of famine.}
        \vspace{7pt}
        \item[b] \textit{Base on Documenting Ireland: Parliament, People and Migration. This article estimates the population in non-census years based on Irish immigration, mortality, and mid-year population data.}
        \vspace{7pt}
        \item[c] \textit{O = Oat, P = Potato, W = Wheat, B = Barley, the following abbreviations are the same}
        \vspace{7pt}
        \item[d] \textit{The potato data in this section are estimated from the agricultural stock situation during this period. Unfortunately, due to the lack of specific yields and the fact that grain yields per hectare are changing, for example, in 1837, the barley yield could reach 24.9 cwt, but the yield from 1847 to 1851 was only 18cwt.}
        \end{spacing}
    \end{TableNotes}
\begin{longtable}{cccc}
    \caption{Data and Sources} \\
    \toprule % 表格顶部线
    \textbf{Data} & \textbf{Details} & \textbf{Time} & \textbf{Sources} \\
    \midrule % 表格标题下方线
    \endfirsthead

    \caption[]{(Continued)} \\
    \toprule
    \textbf{Data} & \textbf{Details} & \textbf{Time} & \textbf{Sources} \\
    \midrule
    \endhead

    \midrule
    \multicolumn{4}{r}{\textit{Continued on next page}} \\
    \midrule
    \endfoot

    \bottomrule % 表格底部线
    \insertTableNotes
    \endlastfoot

    Population & Population & 1821, 1831, \ldots & Irish Census \tnote{a}\\
     & & Remain years & Estimated population \tnote{b}\\
    & & \\
    Wage & General wage & 1821 \textendash\ 1900 & \citep{d1989wages} \& \citep{bishop1915history}\\
    & & \\
    Ground Rent & Ground Rent & 1821 \textendash\ 1829 & \citep{m2013land} \\
     & & 1830 \textendash\ 1849 & \citep{geary2004trends} \\
     & & 1850 \textendash\ 1885 & \citep{guinnane1996bonds} \\
     & & 1886 \textendash\ 1900 & NA \\
    & & \\
    Tax & Tithe & 1821 \textendash\ 1900 & \citep{brynn1970irish} \& \citep{shaw2015economic} \\
    & & \\
    Grain Price & Oat & 1821 \textendash\ 1828 & \citep{daniel2021irish} \\
     & & 1829 \textendash\ 1859 & \citep{vamplew1980grain}\\ 
     & Potato & 1821 \textendash\ 1845 & \citep{kennedy1997prices} \\
     & Wheat & 1824 \textendash\ 1837 & Southampton library\\
     & Barley & 1821 \textendash\ 1828 & \citep{clark2004price} \\
     & O. P. W. B. \tnote{c} & 1840 \textendash\ 1900 & \citep{barrington1926review} \\
     & O. P. & 1821 \textendash\ 1850 & \citep{kennedy1997prices} \\
     & Agriculture index & 1850 \textendash\ 1900 & \citep{turner1987towards}\\
    & & \\

    Plant Acre & Potato & 1821 \textendash\ 1846 & \citep{kenny2023annual} \tnote{d}\\
     & O. W. B. & 1821 \textendash\ 1846 & Estimated from Price Index\\
     & O. P. W. B. & 1847 \textendash\ 1900 & CSO agriculture report \\
    & & \\
    Import & O. W. B. & 1821 \textendash\ 1838 & NA \\
     & O. W. B. & 1839 \textendash\ 1900 & \citep{brunt2004irish} \\
    & & \\
    Export & Wheat & 1821 \textendash\ 1828 & \citep{hansard1840flour} \\
     & O. W. B. & 1829 \textendash\ 1838 & \citep{vamplew1980grain}\\
     & O. W. B. & 1839 \textendash\ 1900 & \citep{brunt2004irish} \\
     & O. B. & 1821 \textendash\ 1828 & NA \\
\end{longtable}
\end{ThreePartTable}
\end{spacing}
\vspace{-14pt}

In addition, there are a number of missing values, including the ground rent from 1886 to 1900, the imports of oats, barley, and wheat from 1821 to 1838, and the exports of barley and oats from 1821 to 1828. Considering that these missing values may be related to other variables, the \texttt{mice} package in \texttt{R} is used to fill these missing values with multiple imputation. Since the data in this article come directly from previous research and historical archives, there is no need to deal with outliers.


\section{Research Hypothesis}

The first part of this paper hypothesizes to focus on the trade-based entitlement, and based on the previous discussion, this entitlement consists of grain prices.

\textbf{$H_1$: A damage in trade-based entitlement, more specifically, an increase in the price of oats, wheat, barley, and potatoes lead to an increase in the number of people who die or migrate during the year.}

And when people received harm on their production-based entitlement, such as an unaffordable tax or land rent, that will also leads to a population decrease.

\textbf{$H_2$: A damage in production-based entitlement, more specifically, an increase in the ground rent and to tax the tithe, lead to an increase in the number of people who die or migrate during the year.}

The third is labor and the rewards received for labor, and, as noted earlier, while farmers' incomes were almost never derived from income compared to citizens, it is also necessary to examine the situation of income.

\textbf{$H_3$: A damage in own-labour entitlement, more specifically, an decrease in the wage, lead to an increase in the number of people who die or migrate during the year.}

Because of the ambiguity of data from the Poor Law, we use the balance of imports and exports to estimate people's entitlement in inheritance and transfer \textemdash\
although the balances maybe not necessarily transferred to the people due to situations like policies, corruption or depletion, etc, it is a hypothesis which is deserve to explore.

\textbf{$H_4$: A damage in inheritance and transfer entitlement, more specifically, an increase in the import and export, lead to an increase in the number of people who die or migrate during the year.}

And a final hypothesis test to refute the FAD theory that population decline should not be blamed on acreage or acreage-related yield issues.

\textbf{$H_{5a/5b}$: There is no relationship between planting acreage and the number of people who die or migrate during the year / There is not enough evidence to suggest that larger planting acreage leads to fewer number of people who die or migrate during the year.}

\section{Statistical Description}
The first step is to perform descriptive statistics on two key variables which are population and the amount of population change in the current year which is calculated by subtracting last year's population from the current year's population and also the latter is the dependent variable for the regression analysis carried out in this paper.

The curve in Figure 3.1 represents the change in population over time, with the lower-middle timeline marking the major historical events of the nineteenth century that had an impact on population, while the bar chart at the bottom records the change in population within the year.

\begin{figure}[htbp]
    \centering
    \caption{Population Index, Decrease and Evens, 1821 \textendash\ 1900}
    \includegraphics[width=\textwidth]{../03_outputs/popline.pdf}
\end{figure}

\begin{figure}[htbp]
    \centering
    \caption{Grain Price, 1821 \textendash\ 1900}
    \includegraphics[width=\textwidth]{../03_outputs/grain_price.pdf}
\end{figure}




\begin{figure}[htbp]
    \centering
    \caption{Regression Scatter}
    \includegraphics[width=.95\textwidth]{../03_outputs/regression_scatter.pdf}
\end{figure}










\chapter{Methods}


\textit{\textendash\ ``Open the gates now. Private, lower your weapon''. \\ 
\textendash\ ``Not till we feed these people. Court martial me, sir. Do whatever you want with me but not till those people are fed''.\\
\textemdash\ ``Black 47'' by Lance Daly}	

\vspace{.2cm}

\section{Generalized Additive Model}

Due to the difficulty in capturing the non-linear relationship, it is necessary to use scatter plots to observe what the non-linear relationship between the independent and dependent variables is. Figure 4.1 provides an overview of the relationship between independent and dependent variables. The three scatter plots in the first column represent three sets of independent and dependent variables with significant linear relationships, while the three scatter plots in the second column represent three sets of non-significant linear, or non-linear, relationships.

Based on the theory of the entitlement approach mentioned earlier, it is indeed possible that there is a non-linear relationship between the independent and dependent variables, for example, when the price of cereals rises marginally, farmers may grow as a result of this profit, whereas when the price of cereals rises significantly, the farmers' trade-base entitlement is consequently jeopardized, and the population of the year ends up declining. According to this logic, linear regression is not a very good choice here and it is necessary to take other forms of non-linear regression for analysis. 

In the scatter plot of Figure 4.1, a non-linear trend can be observed for all three variables in the second column, for example, for wages, it seems that the initial rise was very beneficial for the farmers' production-based entitlement, and as it continued to rise there was a diminishing marginal benefit in economic theory.

\begin{figure}[h]
    \centering
    \caption{Regression Scatter}
    \includegraphics[width=.95\textwidth]{../03_outputs/regression_scatter.pdf}
\end{figure}

Variables in regression included: \texttt{potato\_price}, \texttt{grain\_price\_other}, \texttt{ground\_rent}, \texttt{if\_tithe}, \texttt{general\_wage}, \texttt{poorlaw}, \texttt{imports\_total}, \texttt{exports\_total}. The second regression model includes the variables \texttt{grain\_acre\_total}.

Generalized additive model was used as the main approach in this paper since it is efficient in solving non-linearly relationship from variables by using smooth functions. Based on observation of scatter and correlation matrix, a smoothing function was added to the variables \texttt{grain\_price\_other}, \texttt{general\_wage}, and \texttt{exports\_total}. 

The formulation of the regression model, including the assumptions, is as follows:
\vspace{-14pt}
\begin{align*}
\texttt{E(popgap)} = & \ \beta_0 + \beta_1 \times \texttt{potato\_price} + f_1(\texttt{grain\_price\_other}) \ldots \textit{(H1)} \\
                & + \beta_2 \times \texttt{ground\_rent} + \beta_3 \times \texttt{factor(if\_tithe)} \ldots \ldots .. \textit{(H2)} \\
                & + f_2(\texttt{general\_wage}) \ldots \ldots \ldots \ldots \ldots \ldots \ldots \ldots \ldots \ldots \ldots . \textit{(H3)} \\
                & + \beta_4 \times \texttt{imports\_total} + f_3(\texttt{exports\_total})\\
                & + \beta_6 \times \texttt{factor(poorlaw)} \ldots \ldots \ldots \ldots \ldots \ldots \ldots \ldots \ldots . .. \textit{(H4)} \\
                & + \epsilon
\end{align*}
\vspace{-2cm}
\begin{align*}
\texttt{popgap} = & \ \beta_0 + \beta_1 \times \texttt{grain\_acre\_total}  \ldots \ldots \ldots  \ldots \ldots \ldots \ldots \ldots (\textit{H5a/5b}) \\
& + \epsilon
\end{align*}

Wood demonstrates the usability of the GAM method in  non-linear data and reduces the risk of over-fitting by introducing penalty coefficients in the corresponding calculations \citep{wood2001mgcv}. Recent studies in the demographic have shown that the GAM approach possesses a more significant performance than the GLM approach in fitting regressions to population projections \citep{potts2018evaluation}. In addition, there are also scholars who GAM for entitlement analysis \citep{ardyanto2006granting}, and the regression results also show that GAM can fit the non-linear relationship between entitlements implementation and entitlements consequences well.

This data during famines have always been a focus of debate among different schools, because differences in views will directly lead to the division between nationalism and revisionism \textemdash\ to put it more bluntly, it determines whether scholars will target 19th century's British government. 

\begin{itemize}
    \item [] \textit{``During all the famine year, Ireland actually producing sufficient food, and wool and flax to feed and clothe not nine, but eighteen millions of people''.} \citep{mitchel1905apology}
    \item [] \textit{``At least, historians of Ireland, even the native-born ones, taking them as a group, were not as revisionist in their perspective''.}\citep{donnelly1996construction}
\end{itemize}

Also, the Poor Law is noticed in studies of the famine. The Poor Law and its reform have been mentioned in Chapter 2.1, so I will not repeat it here.

\section{Regression Results Summary}

Firstly is the formula of GAM and FAD LM:
\vspace{-14pt}
\begin{align*}
\texttt{E(popgap)} = & - 7.420 - 0.079 \times \texttt{potato\_price} + f_1(\texttt{grain\_price\_other}) \\
                & + 0.342 \times \texttt{ground\_rent} + \textcolor{red}{0.45 \times \texttt{factor(if\_tithe)}} \\
                & + f_2(\texttt{general\_wage})  \\
                & + 0.044 \times \texttt{imports\_total} + \textcolor{red}{f_3(\texttt{exports\_total})}\\
                & + 3.616 \times \texttt{factor(poorlaw)} \\
                & + \epsilon \\
\texttt{popgap} = & \ 1.987 - 0.003 \times \texttt{grain\_acre\_total} + \epsilon
\end{align*}

The non-significant items have been highlighted in red. All hypotheses are significant for at least one variable, thus allowing for further coefficient interpretation. Table 5.1 lists the values and significance of the coefficients to facilitate comparisons between the model and the hypotheses.

In terms of the trade-based entitlement, an 1 unit increase in the price of potatoes, the population change within a year is associated with an average increase of -7\%; In terms of the production-based entitlement, an 1 unit increase in the ground rent, the population change within a year is associated with an average increase of 34\%; In terms of the inheritance and transfer entitlement, an 1 unit increase in the grain imports, the population change within a year is associated with an average increase of 4\% also the year with poor law will have an average 36\% increase in population change compared with year without poor law; In terms of the rebuttal of the FAD theory, an 1 unit increase in the grain plant acreage, the population change within a year is associated with an average increase of -0.3\%.

\begin{table}[h]
    \centering
    \caption{GAM, FAD LM, and General LM}
    \begin{tabular}{@{\extracolsep{5pt}}lccc}
    \\[-1.8ex]\hline
    \hline \\[-1.8ex]
    & \multicolumn{3}{c}{\textit{Dependent variable: popgap}} \\
    \cline{2-4}
    \\[-1.8ex] & GAM & FAD LM & General LM \\
    \hline \\[-1.8ex]
    potato\_price & $-$0.079$^{***}$ & & $-$0.107$^{***}$ \\
     & (0.028) & & (0.033) \\
    grain\_price\_other & &  & 0.179$^{**}$ \\
     & &  & (0.069) \\
    grain\_acre\_total & & $-$0.003$^{***}$ & \\
     & & (0.001) & \\
    ground\_rent & 0.342$^{***}$ & & 0.082 \\
     & (0.118) & & (0.119) \\
    factor(if\_tithe)1 & 0.450 & & 1.159$^{**}$ \\
     & (0.559) & & (0.535) \\
    general\_wage & & & 0.050 \\
     & & & (0.037) \\
    imports\_total & 0.044$^{***}$ & & 0.042$^{***}$ \\
     & (0.008) & & (0.010) \\
    exports\_total & & & 0.001 \\
     & & & (0.001) \\
    factor(poorlaw)1 & 3.616$^{***}$ & & 3.103$^{***}$ \\
     & (0.617) & & (0.752) \\
    Constant & $-$7.420$^{***}$ & 1.987$^{**}$ & $-$7.751$^{***}$ \\
     & (1.391) & (0.796) & (1.964) \\
    \hline \\[-1.8ex]
    s(grain\_price\_other) & $^{**}$ & & \\
    s(general\_wage) & $^{***}$ & & \\
    s(exports\_total) &  & & \\
    \hline \\[-1.8ex]
    Observations & 80 & 80 & 80 \\
    Adjusted R$^{2}$ & 0.741 & 0.097 & 0.570 \\
    AIC & 201.470 & & 235.916 \\
    Residual Std. Error & & 1.436 (df = 78) & 0.990 (df = 71) \\
    F Statistic & & 9.457$^{***}$ (df = 1; 78) & 14.115$^{***}$ (df = 8; 71) \\
    \hline
    \hline \\[-1.8ex]
    \textit{Note:}  & \multicolumn{3}{r}{$^{*}$p$<$0.1; $^{**}$p$<$0.05; $^{***}$p$<$0.01} \\
    \end{tabular}
\end{table}

Further data-based rebuttals to the FAD theory can be made here. While the linear model coefficients in the FAD theory are significant, i.e., it can be demonstrated that there is a relationship between food supply and demographic change, however, we must pay attention to the magnitude and direction of the coefficients, on the one hand, 0.3\% is a very small coefficient, and its impact on the macro-demographic data needs to be examined; on the other hand, the negative sign in front of the coefficient reveals that in fact there is a negative correlation between food supply and population change, i.e., for 19th Century Ireland, the prevailing trend was that the population was actually decreasing as the supply of food became more plentiful. This negative correlation provides a direct test of \textit{H5}: \textbf{There is not enough evidence to suggest that larger planting acreage lead to an increase in population compared with last year}.

\section{Assumptions and Robustness}

This paper fits two regression model. The first regression model is a GAM model, which is used to prove H1, H2, H3 and H4; the second regression model, due to the linear relationship between variables \texttt{grain\_acre\_total} and \texttt{popgap}, is a linear regression, which is used to prove H5. In fact, for the second regression model, the linear regression and the GAM model have the same AIC, and to follow the modeling principle of simplicity, linear regression is used for fitting.

The necessity and feasibility for the use of the GAM model must be justified before proceeding with the regression analysis. Firstly, the VIF test between the variables shows that there is no multicollinearity between the variables (Table 4.1):

\begin{table}[h] \centering 
    \caption{Model Variance Inflation Factors (VIF)} 
    \label{vif_table} 
    \begin{tabular}{lcccc} 
    \\[-1.8ex]\hline 
    \hline \\[-1.8ex] 
    Variable & VIF & Variable & VIF \\ 
    \hline \\[-1.8ex] 
    potato\_price & 2.044 & general\_wage & 5.390 \\ 
    grain\_price\_other & 1.739 & imports\_total & 6.844 \\ 
    ground\_rent & 3.716 & exports\_total & 1.918 \\ 
    factor(if\_tithe)1 & 5.666 & factor(poorlaw)1 & 4.606 \\ 
    \hline 
    \hline \\[-1.8ex] 
    \end{tabular} 
  \end{table}

Compared to linear regression, the GAM was found to have a significantly higher R-squared and a lower AIC, so it can be concluded that the GAM possesses a better fitting ability and explanatory performance. Figure 4.2 shows the details:

\begin{table}[h] 
    \centering 
    \caption{Regression Results: GAM and Linear} 
  \begin{tabular}{@{\extracolsep{5pt}}lcc} 
  \\[-1.8ex]\hline 
  \hline \\[-1.8ex] 
   & \multicolumn{2}{c}{\textit{Dependent variable:}} \\ 
  \cline{2-3} 
  \\[-1.8ex] & \multicolumn{2}{c}{popgap} \\ 
  \hline \\[-1.8ex] 
   & GAM & LM \\ 
  \hline \\[-1.8ex] 
  \multicolumn{3}{c}{Coefficients are omitted to save space and will be shown in next Chapter} \\
  \hline \\[-1.8ex] 
  Observations & 80 & 80 \\ 
  Adjusted R$^{2}$ & 0.741 & 0.570 \\ 
  AIC & 201.470 & 235.916\\
  Residual Std. Error &  & 0.990 (df = 71) \\ 
  F Statistic &  & 14.115$^{***}$ (df = 8; 71) \\ 
  \hline 
  \hline \\[-1.8ex] 
  \textit{Note:}  & \multicolumn{2}{r}{$^{*}$p$<$0.1; $^{**}$p$<$0.05; $^{***}$p$<$0.01} \\ 
  \end{tabular} 
\end{table} 
The model was tested with \texttt{gam.check()} in \texttt{R}, returning results in Figure 4.2.

\begin{figure}[h]
    \centering
    \caption{Regression Check}
    \includegraphics[width=.9\textwidth]{../03_outputs/regcheck.pdf}
\end{figure}
\vspace{-7pt}

From $P1$, it appears that the distribution of the residuals revolves around the $Y=0$ line with a mean approximately equal to 0 and no pattern can be found; whereas the Q-Q plot of $P2$ indicates a normal distribution structure of the data; $P3$ also shows a uniform and random distribution between the response value and fitting value; and finally, the residuals indicated by $P4$ show a normal distribution.


\section{Robustness Test}

Robustness tests are performed on the most dominant model of the paper, i.e., the GAM that incorporates the four hypotheses.

\begin{table}[h]
    \centering
    \caption{Robustness Test}
    \begin{tabular}{@{\extracolsep{5pt}}lccccc}
    \\[-1.8ex]\hline
    \hline \\[-1.8ex]
    & \multicolumn{5}{c}{\textit{Dependent variable: popgap}} \\
    \cline{2-6}
    \\[-1.8ex] & GAM 1 & LM & GAM 2 & GAM 3 & GAM 4 \\
    & \textit{standard} & \textit{linear} & \textit{year} & \textit{half year} & \textit{norm.} \\
    \hline \\[-1.8ex]
    potato\_price & $-$0.079$^{***}$ & $-$0.107$^{***}$ & $-$0.096$^{**}$ & $-$0.133$^{**}$ & $-$0.063$^{*}$ \\
     & (0.028) & (0.033) & (0.028) & (0.050) & (0.025) \\
    grain\_price\_other & & 0.178$^{*}$ &  & & \\
     & & (0.069) &  & & \\
    ground\_rent & 0.342$^{***}$ & 0.082 & 0.306$^{*}$ & 0.266$^{.}$ & 0.323$^{**}$ \\
     & (0.118) & (0.119) & (0.123) & (0.155) & (0.111) \\
    factor(if\_tithe)1 & 0.450 & 1.159$^{*}$ & 0.905 & 1.999$^{.}$ & 0.924$^{.}$ \\
     & (0.559) & (0.535) & (0.610) & (1.086) & (0.509) \\
    general\_wage & & 0.050 & & 1.021$^{***}$ & 1.021$^{***}$ \\
     & & (0.037) & & (0.261) & (0.261) \\
    imports\_total & 0.044$^{***}$ & 0.042$^{***}$ & 0.032$^{***}$ & 0.049$^{***}$ & 0.056$^{***}$ \\
     & (0.008) & (0.010) & (0.009) & (0.012) & (0.008) \\
    exports\_total & & 0.001 & & & \\
     & & (0.001) & & & \\
    factor(poorlaw)1 & 3.616$^{***}$ & 3.103$^{***}$ & 2.867$^{***}$ & 3.832$^{***}$ & 3.812$^{***}$ \\
     & (0.617) & (0.752) & (0.664) & (0.901) & (0.523) \\
    \hline \\[-1.8ex]
    s(grain\_price\_other) & $^{**}$ & & $^{**}$ & $^{*}$ & $^{***}$  \\
    s(general\_wage) & $^{***}$ & & $^{***}$ & & $^{***}$ \\
    s(exports\_total) & & & \\
    \hline \\[-1.8ex]
    Observations & 80 & 80 & 80 & 40 & 80 \\
    \hline
    \hline \\[-1.8ex]
    \textit{Note:} & \multicolumn{5}{r}{$^{*}$p$<$0.1; $^{**}$p$<$0.05; $^{***}$p$<$0.01} \\
    \end{tabular}
\end{table}

\section{Brief Summary}

An overview of the research logic used throughout the article is given.

First is the theoretical framework: (1) entitlement approach used by Amartya Sen in his book ``Poverty and Famine'', and (2) refutation of the FAD theory. According to Sen's construct, the entitlement approach consists of the following four indicators: trade-based entitlement, production-based entitlement, own-labour entitlement and Inheritance and transfer entitlement; while the FAD theory includes one indicator, the area under food cultivation. This paper generalizes the entitlement approach as a demographic and developmental mechanism based on the literature and attempts to explore if the changes in people's entitlements, which is the independent variable, will affect population change within the year, which is the dependent variable.

Further, hypotheses are made on the basis of these indicators, with each indicator corresponding to a hypothesis and each hypothesis corresponding to a more specific variable in the dataset. $H1$ corresponds to price of potatoes and other cereals, $H2$ corresponds to ground rent and presence or absence of tithing, $H3$ corresponds to general wage, $H4$ corresponds to the imports and exports amount, and the Poor Law, and $H5$ corresponds to the grain planting acreage.

The GAM was used in this study for a number of reasons, including (1) there is no significant linear relationship between some independent variables and the dependent variable, but there is an observable nonlinear relationship and theory supports the existence of such a nonlinear relationship, and (2) the introduction of a smoothing function in the GAM can predict this nonlinear relationship in a good way. Relative tests have been performed before formally discussing the regression, including the VIF test for multicollinearity, AIC and R-square comparisons for linear regression, the residual randomness test, and the residual normality test. All of the results identify the GAM as a reasonable regression model under this data.

Figure 4.3 visualizes the framework of the entire research.

\begin{landscape}
    \begin{figure}[h]
        \centering
        \caption{Research Framework}
        \includegraphics[width=1.5\textheight]{../03_outputs/Framework.pdf}
    \end{figure}
\end{landscape}




\section{Replication}
All the replication file can be found in this website:

\url{https://github.com/chxiii/Dissertation_Summer2024}




\chapter{Empirical Analysis}

\textit{``\textendash\ Those dying generations \textendash\ at their song,\\
The salmon-falls, the mackerel-crowded seas,\\
Fish, flesh, or fowl, commend all summer long\\
Whatever is begotten, born, and dies''.\\
\textemdash\ ``Sailing to Byzantium'' by William Butler Yeats }

\vspace{.2cm}


\begin{figure}[h]
    \centering
    \caption{Significant Smooth Term Plot}
    \includegraphics[width=.95\textwidth]{../03_outputs/smoothterm.pdf}
\end{figure}




\chapter{Conclusion}

\textit{
    ``ESTRAGON: (looking at the tree). What is it? \\
    VLADIMIR: It's the tree. \\
    ESTRAGON: Yes, but what kind? \\
    VLADIMIR: I don't know. A willow. \\
    Estragon draws Vladimir towards the tree. They stand 
    motionless before it. Silence. \\
    \textemdash\ ``Waiting for Godot'' by Samuel Beckett
} \citep{beck1982waiting}

This research use Amartya's entitlement approach to focus on famine, and then the population change during the 19th century in Ireland. Unlike typical famine and population research, which doing the research under a FAD theory framework, this research rebut it first. During quantitative progress, these arguments are drew: 

Firstly, FAD is not suitable for Irish case although potato blight happened. Using grain import and export amount as control variables, and grain planting acreage as key independent variable, this paper conduct a LM regression and shows a non-significant coefficient in grain acreage, which shows there is no relationship between grain acreage and population change in the 19th century Ireland.

Secondly, entitlement approach, both on the effect of famine and of the population development, is an important mechanism to explain. With the same control variables, and grain price, ground rent, tax status, wage and poor law as key independent variables, this paper conduct a GAM regression and shows a significant coefficient in potato price, wage and poor law, which prove the trade-based entitlement, own-labour entitlement and transfer and inheritance entitlement are making influence in the population change during the 19th century Ireland respectfully.

\newpage
\addcontentsline{toc}{chapter}{References}
\bibliographystyle{agsm}
\bibliography{reference}

\appendix
\renewcommand{\thechapter}{A\arabic{chapter}}

\end{document}