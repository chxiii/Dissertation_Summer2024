\chapter{Literature Review}

\textit{``Hunger roared up in him like a hopeless lust.\\ 
He walked the ship as though following a chart. Up. Down. Across. Back. Stem. Port. Stern. Starboard. The churning of the waves. \\
The ropes clanking on the masts. The blind of salt water. The wind ripping at the sails.''\\
\textemdash\ ``Star of the Sea'' by Joseph O'Connor}
\vspace{.2cm}

\section{A Brief Famine Outline}

The Irish lumper potato with its excellent ability to grow in poor and wet soils, was the predominant potato variety in pre-famine Ireland. It was introduced to U.K. around 1806 \citep{tucker2016potato}, and rapidly replacing almost all other varieties in the recipes of the poor. Usually, on account of its intolerance of frost, the farmer sows in March or April, and the first early potatoes will be harvested in June, followed by the second early potatoes in July, and the third not later than October. With a 1.32 \% growth in lower class per year in Ireland from the centennial before 1841, in 1845 about 32\% of the arable land in Ireland was already under potato cultivation \citep{solar2015ireland}.

The first record of late blight on potatoes in Ireland is thought to be Dr Lindley's letter in September 16, 1845, with his concern words, he wrote: ``The potato murrain has unequivocally declared itself in Ireland, where will Ireland be in the event of a universal potato rot''? \citep{kelly1995great}. Things were getting worse in 1846, a government documents collection book recorded that: ``the poor Irish lost their potatoes again'' (1 September, 1846) so that ``Many, full many, must this winter leave their homes, and traverse the country in quest of work'' (15 September, 1846). Government employee pointed out a fact, ``to maintain Ireland's population, her agriculture must be greatly improved'' (31 October, 1846). Next year, due to a change in the Poor Law, ``the poorest peasantry were draught to the shore of America'' (18 January, 1847), but didn't seems to release the effect of famine. Later, in newspaper's leading article, reporter wrote: ``eye-witnesses of scores and hundreds of poor creatures actually dying for want a meal'' (8 March, 1847) and all ``landlord, tenure and peasant were in a miserable situation'' (13 March, 1847). Reflection was raising and people started to realized a serious famine come back since 1741 because ``the food that suffered in both years was the same'' (14 April, 1847). Till November, the exodus of the population was getting worse and caused the ``disorder in Ireland'' (November 13, 1847). Finally, because of sharply decrease population, Ireland faced a situation ``Labour is the first price'' (December 30, 1847) \citep{times1880famineletter}.

Throughout the history of the famine and pre-famine period, the role of the Poor Law cannot be ignored. The Poor Law was introduced in Ireland in July 1838 with the blueprint of the Poor Law in England and Wales, and provided for the establishment of 130 trade unions throughout Ireland, where the poor were to be relieved and regulated by the guardians of the trade unions \citep{o1985new}. However, in January 1847, the government pushed for reform of the Poor Law, which exacerbated the ravages of famine in Ireland \textendash\ particularly in the south and west of Ireland. The most significant consequence of the reforms was the almost complete transfer of responsibility and financial pressure for poverty alleviation to local government finances, which in the context of the famine resulted in the complete collapse of the local poverty alleviation system. It is very difficult to objectively assess the role of poverty law, which on the one hand does provide relief to many poor people \citep{mchugh1986famine}, but on the other hand is also characterized by Foucault's theory of power genealogy like ``micro-power'' and the operation of ``bio-politics'', as the 1847 letter reads:

\begin{itemize}
    \item [] \textit{It is true we have been careful not to put forward a poor-law as a mean to supply, but have claimed for it only a place among the means of distributing supplies \textendash\ of promoting employment, and of enforcing upon poverty the care and protection of the labour. Still, if that surplus of unfilled mouths is to be always in front of us, it must be confessed that very little good, after all, will be accomplished.} \citep{thomas1847poorlaw}
\end{itemize}

After 1847, the rate of depopulation slowed and the most difficult period was over. Some scholars have pointed out that the cause of death of the population during this period was more due to diseases brought about by the famine, including dysentery, diarrhea, tuberculosis, fever, and swelling \citep{mokyr2002people}. The 1851 census showed the population declined by approximately 1.62 million after the famine.

From the census data of 1841 and 1851, we can calculate the change in population of the different provinces after the famine, which showed the result that the west and south suffered far more from famine than the east and north  (Figure 2.1). The five counties with the greatest decreases in population are Donegal, Connaught, in the west, 279,601; Cork, Munster, in the south, 209,822; Galway, Connaught, in the west, 125,026; Tipperary, Munster, in the south, 103,986, and Roscommon, Connaught, in the west, 80,155.\ \textit{The freeman's journal} similarly supports this conclusion in its April 27, 1847 article documenting the damage to parishes including Killedy, Toomavara, Abbey, Lorha and Dorrow, etc \citep{freeman1847parishes}. 

\begin{figure}[htbp]
    \centering
    \caption{County Population 1841 \textendash\ 1851}
    \includegraphics[width=\textwidth]{../03_outputs/map1841_1851.pdf}
\end{figure}

\section{Rebut Food Availability Decline (FAD) Theory}

As Amartya put it in his book \textendash\ ``\textit{the most common approach to famines is to propose explanations in terms of food availability decline (FAD)}'' \citep{sen1982poverty} \textendash\ In the Irish famine, where the FAD theory contains two aspects: (1) the potato late blight; (2) the monocultural structure of the Irish diet. For a long time it was believed that these two were at the root of the famine, in historical fact, however, while the existence of both is undeniable, their impact is not decisive.

Firstly, potato late blight. During the middle 19th century, 
people saw the potato blight, they saw the famine, which made they think potato late blight was directly associated with famine due to the empirical observations and logical extrapolations. Native Irish farmers have a set of folk myths about this, believing that fairies in the sky were fighting over the potatoes, or that Fear Liath, the fog man, which led to the blight and famine \citep{bartoletti2001black}. Also, references in correspondence with the British government mentioned the relationship between late blight and death in potatoes:

\textit{``In the year 1845, I think about the month of July or the beginning of August, the potatoes withered and decayed all over the country like what you have seen on
the watersides with early frost, [\ldots] poor families were badly off and striving to live on bran.''} \citep{mcclureletter1848}

\textit{``When they came back home, there was not a potato in what they dug but was infected [\ldots], it is the whole cry among the people.''} \citep{blackwellletter1845}



Braa, on the other hand, argues that the potato blight was as important as a single dietary structure in influencing the Irish famine.\citep{braa1997great}

This part I will refute some hypothesis of famine origin. Many people regard single factor as the root of the Great Famine.

\begin{figure}[htbp]
    \centering
    \caption{Grain Agriculture Structure 1820 \textendash\ 1900}
    \includegraphics[width=.95\textwidth]{../03_outputs/food_structure.pdf}
\end{figure}

In Nature journal,

1845 June Belgium, August France, August South of UK, September Ireland

\begin{figure}[htbp]
    \centering
    \caption{Grain Price 1820 \textendash\ 1900}
    \includegraphics[width=.95\textwidth]{../03_outputs/grain_price.pdf}
\end{figure}



1. Blame potato blight as the only origin of famine

People believe potato blight was responsible for the Irish Great Famine. 

lumper potato

Blight became a semi-permanent fixture until the end of the century, when effective treatments were found \citep{o1994economic}.

2. Ireland have the bad land quality.

\section{Entitlement Approach}



I n the field of famine studies, scholars as diverse asSusan George (1980), Amartya Sen (1981, 2000),Michael Watts (1983), Amrita Rangasami (1985),and Stephen Devereux (2001) have argued that faminesdo not necessarily begin with crop failures, droughts, orequivalent climatic hazards. On the contrary, their vi-olence is coordinated much earlier when a populationis progressively brought to the point of collapse. Readthis way, a crop failure, or indeed a drought, is simply an“environmental trigger” in a much larger narrative of ag-gregated poverty and mass vulnerability (George 1984;Devereux 2002). Despite the fact that the Great IrishFamine is now a major field of scholarly enquiry, therehas been very little attempt to engage with these criti-cal perspectives—derived primarily from famine expe-riences in the global South—nor has there been any at-tempt to analyze the Great Famine from the perspectiveof colonial governance and population management. \citep{nally2008coming}


I will operationalize entitlement approach into these 4 dimensions according to the book:

(1) trade-based entitlement: price, grain amount, 

(2) production-based entitlement: tax policy

(3) own-labour entitlement: wage, land own amount, poor law

(4) inheritance and transfer entitlement: none, hard to get data






