\chapter{Methods}


\textit{\textendash\ ``Open the gates now. Private, lower your weapon''. \\ 
\textendash\ ``Not till we feed these people. Court martial me, sir. Do whatever you want with me but not till those people are fed''.\\
\textemdash\ ``Black 47'' by Lance Daly}	

\vspace{.2cm}

\section{Generalized Additive Model}

Due to the difficulty in capturing the non-linear relationship, it is necessary to use scatter plots to observe what the non-linear relationship between the independent and dependent variables is. Figure 4.1 provides an overview of the relationship between independent and dependent variables. The three scatter plots in the first column represent three sets of independent and dependent variables with significant linear relationships, while the three scatter plots in the second column represent three sets of non-significant linear, or non-linear, relationships.

Based on the theory of the entitlement approach mentioned earlier, it is indeed possible that there is a non-linear relationship between the independent and dependent variables, for example, when the price of cereals rises marginally, farmers may grow as a result of this profit, whereas when the price of cereals rises significantly, the farmers' trade-base entitlement is consequently jeopardized, and the population of the year ends up declining. According to this logic, linear regression is not a very good choice here and it is necessary to take other forms of non-linear regression for analysis. 

In the scatter plot of Figure 4.1, a non-linear trend can be observed for all three variables in the second column, for example, for wages, it seems that the initial rise was very beneficial for the farmers' production-based entitlement, and as it continued to rise there was a diminishing marginal benefit in economic theory.

\begin{figure}[h]
    \centering
    \caption{Regression Scatter}
    \includegraphics[width=.95\textwidth]{../03_outputs/regression_scatter.pdf}
\end{figure}

Variables in regression included: \texttt{potato\_price}, \texttt{grain\_price\_other}, \texttt{ground\_rent}, \texttt{if\_tithe}, \texttt{general\_wage}, \texttt{poorlaw}, \texttt{imports\_total}, \texttt{exports\_total}. The second regression model includes the variables \texttt{grain\_acre\_total}.

Generalized additive model was used as the main approach in this paper since it is efficient in solving non-linearly relationship from variables by using smooth functions. Based on observation of scatter and correlation matrix, a smoothing function was added to the variables \texttt{grain\_price\_other}, \texttt{general\_wage}, and \texttt{exports\_total}. 

The formulation of the regression model, including the assumptions, is as follows:
\vspace{-14pt}
\begin{align*}
\texttt{E(popgap)} = & \ \beta_0 + \beta_1 \times \texttt{potato\_price} + f_1(\texttt{grain\_price\_other}) \ldots \textit{(H1)} \\
                & + \beta_2 \times \texttt{ground\_rent} + \beta_3 \times \texttt{factor(if\_tithe)} \ldots \ldots .. \textit{(H2)} \\
                & + f_2(\texttt{general\_wage}) \ldots \ldots \ldots \ldots \ldots \ldots \ldots \ldots \ldots \ldots \ldots . \textit{(H3)} \\
                & + \beta_4 \times \texttt{imports\_total} + f_3(\texttt{exports\_total})\\
                & + \beta_6 \times \texttt{factor(poorlaw)} \ldots \ldots \ldots \ldots \ldots \ldots \ldots \ldots \ldots . .. \textit{(H4)} \\
                & + \epsilon
\end{align*}
\vspace{-2cm}
\begin{align*}
\texttt{popgap} = & \ \beta_0 + \beta_1 \times \texttt{grain\_acre\_total}  \ldots \ldots \ldots  \ldots \ldots \ldots \ldots \ldots (\textit{H5a/5b}) \\
& + \epsilon
\end{align*}

Wood demonstrates the usability of the GAM method in  non-linear data and reduces the risk of over-fitting by introducing penalty coefficients in the corresponding calculations \citep{wood2001mgcv}. Recent studies in the demographic have shown that the GAM approach possesses a more significant performance than the GLM approach in fitting regressions to population projections \citep{potts2018evaluation}. In addition, there are also scholars who GAM for entitlement analysis \citep{ardyanto2006granting}, and the regression results also show that GAM can fit the non-linear relationship between entitlements implementation and entitlements consequences well.

This data during famines have always been a focus of debate among different schools, because differences in views will directly lead to the division between nationalism and revisionism \textemdash\ to put it more bluntly, it determines whether scholars will target 19th century's British government. 

\begin{itemize}
    \item [] \textit{``During all the famine year, Ireland actually producing sufficient food, and wool and flax to feed and clothe not nine, but eighteen millions of people''.} \citep{mitchel1905apology}
    \item [] \textit{``At least, historians of Ireland, even the native-born ones, taking them as a group, were not as revisionist in their perspective''.}\citep{donnelly1996construction}
\end{itemize}

Also, the Poor Law is noticed in studies of the famine. The Poor Law and its reform have been mentioned in Chapter 2.1, so I will not repeat it here.

\section{Regression Results Summary}

Firstly is the formula of GAM and FAD LM:
\vspace{-14pt}
\begin{align*}
\texttt{E(popgap)} = & - 7.420 - 0.079 \times \texttt{potato\_price} + f_1(\texttt{grain\_price\_other}) \\
                & + 0.342 \times \texttt{ground\_rent} + \textcolor{red}{0.45 \times \texttt{factor(if\_tithe)}} \\
                & + f_2(\texttt{general\_wage})  \\
                & + 0.044 \times \texttt{imports\_total} + \textcolor{red}{f_3(\texttt{exports\_total})}\\
                & + 3.616 \times \texttt{factor(poorlaw)} \\
                & + \epsilon \\
\texttt{popgap} = & \ 1.987 - 0.003 \times \texttt{grain\_acre\_total} + \epsilon
\end{align*}

The non-significant items have been highlighted in red. All hypotheses are significant for at least one variable, thus allowing for further coefficient interpretation. Table 5.1 lists the values and significance of the coefficients to facilitate comparisons between the model and the hypotheses.

In terms of the trade-based entitlement, an 1 unit increase in the price of potatoes, the population change within a year is associated with an average increase of -7\%; In terms of the production-based entitlement, an 1 unit increase in the ground rent, the population change within a year is associated with an average increase of 34\%; In terms of the inheritance and transfer entitlement, an 1 unit increase in the grain imports, the population change within a year is associated with an average increase of 4\% also the year with poor law will have an average 36\% increase in population change compared with year without poor law; In terms of the rebuttal of the FAD theory, an 1 unit increase in the grain plant acreage, the population change within a year is associated with an average increase of -0.3\%.

\begin{table}[h]
    \centering
    \caption{GAM, FAD LM, and General LM}
    \begin{tabular}{@{\extracolsep{5pt}}lccc}
    \\[-1.8ex]\hline
    \hline \\[-1.8ex]
    & \multicolumn{3}{c}{\textit{Dependent variable: popgap}} \\
    \cline{2-4}
    \\[-1.8ex] & GAM & FAD LM & General LM \\
    \hline \\[-1.8ex]
    potato\_price & $-$0.079$^{***}$ & & $-$0.107$^{***}$ \\
     & (0.028) & & (0.033) \\
    grain\_price\_other & &  & 0.179$^{**}$ \\
     & &  & (0.069) \\
    grain\_acre\_total & & $-$0.003$^{***}$ & \\
     & & (0.001) & \\
    ground\_rent & 0.342$^{***}$ & & 0.082 \\
     & (0.118) & & (0.119) \\
    factor(if\_tithe)1 & 0.450 & & 1.159$^{**}$ \\
     & (0.559) & & (0.535) \\
    general\_wage & & & 0.050 \\
     & & & (0.037) \\
    imports\_total & 0.044$^{***}$ & & 0.042$^{***}$ \\
     & (0.008) & & (0.010) \\
    exports\_total & & & 0.001 \\
     & & & (0.001) \\
    factor(poorlaw)1 & 3.616$^{***}$ & & 3.103$^{***}$ \\
     & (0.617) & & (0.752) \\
    Constant & $-$7.420$^{***}$ & 1.987$^{**}$ & $-$7.751$^{***}$ \\
     & (1.391) & (0.796) & (1.964) \\
    \hline \\[-1.8ex]
    s(grain\_price\_other) & $^{**}$ & & \\
    s(general\_wage) & $^{***}$ & & \\
    s(exports\_total) &  & & \\
    \hline \\[-1.8ex]
    Observations & 80 & 80 & 80 \\
    Adjusted R$^{2}$ & 0.741 & 0.097 & 0.570 \\
    AIC & 201.470 & & 235.916 \\
    Residual Std. Error & & 1.436 (df = 78) & 0.990 (df = 71) \\
    F Statistic & & 9.457$^{***}$ (df = 1; 78) & 14.115$^{***}$ (df = 8; 71) \\
    \hline
    \hline \\[-1.8ex]
    \textit{Note:}  & \multicolumn{3}{r}{$^{*}$p$<$0.1; $^{**}$p$<$0.05; $^{***}$p$<$0.01} \\
    \end{tabular}
\end{table}

Further data-based rebuttals to the FAD theory can be made here. While the linear model coefficients in the FAD theory are significant, i.e., it can be demonstrated that there is a relationship between food supply and demographic change, however, we must pay attention to the magnitude and direction of the coefficients, on the one hand, 0.3\% is a very small coefficient, and its impact on the macro-demographic data needs to be examined; on the other hand, the negative sign in front of the coefficient reveals that in fact there is a negative correlation between food supply and population change, i.e., for 19th Century Ireland, the prevailing trend was that the population was actually decreasing as the supply of food became more plentiful. This negative correlation provides a direct test of \textit{H5}: \textbf{There is not enough evidence to suggest that larger planting acreage lead to an increase in population compared with last year}.

\section{Assumptions and Robustness}

This paper fits two regression model. The first regression model is a GAM model, which is used to prove H1, H2, H3 and H4; the second regression model, due to the linear relationship between variables \texttt{grain\_acre\_total} and \texttt{popgap}, is a linear regression, which is used to prove H5. In fact, for the second regression model, the linear regression and the GAM model have the same AIC, and to follow the modeling principle of simplicity, linear regression is used for fitting.

The necessity and feasibility for the use of the GAM model must be justified before proceeding with the regression analysis. Firstly, the VIF test between the variables shows that there is no multicollinearity between the variables (Table 4.1):

\begin{table}[h] \centering 
    \caption{Model Variance Inflation Factors (VIF)} 
    \label{vif_table} 
    \begin{tabular}{lcccc} 
    \\[-1.8ex]\hline 
    \hline \\[-1.8ex] 
    Variable & VIF & Variable & VIF \\ 
    \hline \\[-1.8ex] 
    potato\_price & 2.044 & general\_wage & 5.390 \\ 
    grain\_price\_other & 1.739 & imports\_total & 6.844 \\ 
    ground\_rent & 3.716 & exports\_total & 1.918 \\ 
    factor(if\_tithe)1 & 5.666 & factor(poorlaw)1 & 4.606 \\ 
    \hline 
    \hline \\[-1.8ex] 
    \end{tabular} 
  \end{table}

Compared to linear regression, the GAM was found to have a significantly higher R-squared and a lower AIC, so it can be concluded that the GAM possesses a better fitting ability and explanatory performance. Figure 4.2 shows the details:

\begin{table}[h] 
    \centering 
    \caption{Regression Results: GAM and Linear} 
  \begin{tabular}{@{\extracolsep{5pt}}lcc} 
  \\[-1.8ex]\hline 
  \hline \\[-1.8ex] 
   & \multicolumn{2}{c}{\textit{Dependent variable:}} \\ 
  \cline{2-3} 
  \\[-1.8ex] & \multicolumn{2}{c}{popgap} \\ 
  \hline \\[-1.8ex] 
   & GAM & LM \\ 
  \hline \\[-1.8ex] 
  \multicolumn{3}{c}{Coefficients are omitted to save space and will be shown in next Chapter} \\
  \hline \\[-1.8ex] 
  Observations & 80 & 80 \\ 
  Adjusted R$^{2}$ & 0.741 & 0.570 \\ 
  AIC & 201.470 & 235.916\\
  Residual Std. Error &  & 0.990 (df = 71) \\ 
  F Statistic &  & 14.115$^{***}$ (df = 8; 71) \\ 
  \hline 
  \hline \\[-1.8ex] 
  \textit{Note:}  & \multicolumn{2}{r}{$^{*}$p$<$0.1; $^{**}$p$<$0.05; $^{***}$p$<$0.01} \\ 
  \end{tabular} 
\end{table} 
The model was tested with \texttt{gam.check()} in \texttt{R}, returning results in Figure 4.2.

\begin{figure}[h]
    \centering
    \caption{Regression Check}
    \includegraphics[width=.9\textwidth]{../03_outputs/regcheck.pdf}
\end{figure}
\vspace{-7pt}

From $P1$, it appears that the distribution of the residuals revolves around the $Y=0$ line with a mean approximately equal to 0 and no pattern can be found; whereas the Q-Q plot of $P2$ indicates a normal distribution structure of the data; $P3$ also shows a uniform and random distribution between the response value and fitting value; and finally, the residuals indicated by $P4$ show a normal distribution.


\section{Robustness Test}

Robustness tests are performed on the most dominant model of the paper, i.e., the GAM that incorporates the four hypotheses.

\begin{table}[h]
    \centering
    \caption{Robustness Test}
    \begin{tabular}{@{\extracolsep{5pt}}lccccc}
    \\[-1.8ex]\hline
    \hline \\[-1.8ex]
    & \multicolumn{5}{c}{\textit{Dependent variable: popgap}} \\
    \cline{2-6}
    \\[-1.8ex] & GAM 1 & LM & GAM 2 & GAM 3 & GAM 4 \\
    & \textit{standard} & \textit{linear} & \textit{year} & \textit{half year} & \textit{norm.} \\
    \hline \\[-1.8ex]
    potato\_price & $-$0.079$^{***}$ & $-$0.107$^{***}$ & $-$0.096$^{**}$ & $-$0.133$^{**}$ & $-$0.063$^{*}$ \\
     & (0.028) & (0.033) & (0.028) & (0.050) & (0.025) \\
    grain\_price\_other & & 0.178$^{*}$ &  & & \\
     & & (0.069) &  & & \\
    ground\_rent & 0.342$^{***}$ & 0.082 & 0.306$^{*}$ & 0.266$^{.}$ & 0.323$^{**}$ \\
     & (0.118) & (0.119) & (0.123) & (0.155) & (0.111) \\
    factor(if\_tithe)1 & 0.450 & 1.159$^{*}$ & 0.905 & 1.999$^{.}$ & 0.924$^{.}$ \\
     & (0.559) & (0.535) & (0.610) & (1.086) & (0.509) \\
    general\_wage & & 0.050 & & 1.021$^{***}$ & 1.021$^{***}$ \\
     & & (0.037) & & (0.261) & (0.261) \\
    imports\_total & 0.044$^{***}$ & 0.042$^{***}$ & 0.032$^{***}$ & 0.049$^{***}$ & 0.056$^{***}$ \\
     & (0.008) & (0.010) & (0.009) & (0.012) & (0.008) \\
    exports\_total & & 0.001 & & & \\
     & & (0.001) & & & \\
    factor(poorlaw)1 & 3.616$^{***}$ & 3.103$^{***}$ & 2.867$^{***}$ & 3.832$^{***}$ & 3.812$^{***}$ \\
     & (0.617) & (0.752) & (0.664) & (0.901) & (0.523) \\
    \hline \\[-1.8ex]
    s(grain\_price\_other) & $^{**}$ & & $^{**}$ & $^{*}$ & $^{***}$  \\
    s(general\_wage) & $^{***}$ & & $^{***}$ & & $^{***}$ \\
    s(exports\_total) & & & \\
    \hline \\[-1.8ex]
    Observations & 80 & 80 & 80 & 40 & 80 \\
    \hline
    \hline \\[-1.8ex]
    \textit{Note:} & \multicolumn{5}{r}{$^{*}$p$<$0.1; $^{**}$p$<$0.05; $^{***}$p$<$0.01} \\
    \end{tabular}
\end{table}

\section{Brief Summary}

An overview of the research logic used throughout the article is given.

First is the theoretical framework: (1) entitlement approach used by Amartya Sen in his book ``Poverty and Famine'', and (2) refutation of the FAD theory. According to Sen's construct, the entitlement approach consists of the following four indicators: trade-based entitlement, production-based entitlement, own-labour entitlement and Inheritance and transfer entitlement; while the FAD theory includes one indicator, the area under food cultivation. This paper generalizes the entitlement approach as a demographic and developmental mechanism based on the literature and attempts to explore if the changes in people's entitlements, which is the independent variable, will affect population change within the year, which is the dependent variable.

Further, hypotheses are made on the basis of these indicators, with each indicator corresponding to a hypothesis and each hypothesis corresponding to a more specific variable in the dataset. $H1$ corresponds to price of potatoes and other cereals, $H2$ corresponds to ground rent and presence or absence of tithing, $H3$ corresponds to general wage, $H4$ corresponds to the imports and exports amount, and the Poor Law, and $H5$ corresponds to the grain planting acreage.

The GAM was used in this study for a number of reasons, including (1) there is no significant linear relationship between some independent variables and the dependent variable, but there is an observable nonlinear relationship and theory supports the existence of such a nonlinear relationship, and (2) the introduction of a smoothing function in the GAM can predict this nonlinear relationship in a good way. Relative tests have been performed before formally discussing the regression, including the VIF test for multicollinearity, AIC and R-square comparisons for linear regression, the residual randomness test, and the residual normality test. All of the results identify the GAM as a reasonable regression model under this data.

Figure 4.3 visualizes the framework of the entire research.

\begin{landscape}
    \begin{figure}[h]
        \centering
        \caption{Research Framework}
        \includegraphics[width=1.5\textheight]{../03_outputs/Framework.pdf}
    \end{figure}
\end{landscape}




\section{Replication}
All the replication file can be found in this website:

\url{https://github.com/chxiii/Dissertation_Summer2024}



