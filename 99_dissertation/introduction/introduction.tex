\chapter{Introduction}
\textit{``October playing a symphony on a slack wire paling.\\
Maguire watches the drills flattened out\\
And the flints that lit a candle for him on a June altar,\\
Flameless''. \\
\textemdash\ ``The Great Hunger'' by Patrick Kavanagh.} \citep{Kavanagh_Quinn_2006a}
\vspace{.2cm}

The Irish Great Famine (1845 \textendash\ 1851) reshaped the entire history of Ireland. Before the Great Famine, according to the 1841 census, the population of the Ireland had close to 8.5 million
\footnote{
	\href{https://www.cso.ie/en/statistics/historicalreports/census1841/}{1841 Census of Ireland, Last accessed: 13 May, 2024}
}
. In 1851, when the Irish Great Famine had not yet ended, census noted that about 1 million people had died for hunger, and a similar number had gone into overseas exile
\footnote{
	\href{https://www.cso.ie/en/statistics/historicalreports/census1851/}{1851 Census of Ireland, Last accessed: 2 May, 2024}
}. In 1926, as a result of the Irish independence 5 years earlier, the Central Statistical Office was capable to integrate historical documents since famine and showed the fact that the population was decline of roughly 22\%
\footnote{
	\href{https://www.cso.ie/en/media/csoie/census/census1926results/volume10/C_1926_V10_Chapter_II.pdf}{1926 Census of Ireland, Chapter II, Last acceseed: 9 May, 2024}
} in the 10 years from 1841 to 1851. Using parish baptism data, some scholars have estimated that in the year 1847 alone \textendash\ which is also known as black'47 in Ireland history \textendash\ there existed counties with a nearly 70\% reduction in baptisms in Munster province in the south of Ireland \citep{cousens1960regional}, especially from southwest Cork and including north and east Clare
\footnote{
	\href{https://www.rte.ie/history/the-great-irish-famine/2020/0629/1150367-the-great-irish-famine/}{RTE, How "a truly modern famine" devastated Ireland, Last accessed: 11 May, 2024}
}
, while it was not the worst hit by the famine compared to the province of Connacht in the west
\footnote{
	\href{https://www.wesleyjohnston.com/users/ireland/past/famine/summer_1847.html}{Wesley Johnston: The Famine: The Summer of Black'47, Last accessed: 13 May, 2024}
}
. Apart from these quantitative explorations, the Great Famine is equally pivotal in Irish cultural history and ethnography. From Joseph O'Connor's fiction ``Star of the sea'' to W. B. Yeats's ``The Countess Cathleen'', together they expressed that the Great Famine not only pointed to the corpses of the dead, but also to a black hole of identity, naming and meaning \citep{luchen41naming}. 

The effects of the Great Famine were far-reaching, and reflected in the long-term population development, land institution structure and attitude to the UK government directly. It was not until 120 years later, in the 1960s, that Ireland's population began to grow consistently due to large-scale emigration, late marriage and a high incidence of permanent celibacy no longer hold \citep{grada1979population}, but it was still nowhere near as large as it had been during the Great Famine
\footnote{
	\href{https://www.cso.ie/en/releasesandpublications/ep/p-cpsr/censusofpopulation2022-summaryresults/populationchanges/}
	{2022 Census of Ireland \textendash\ Summary Results, Last accessed: 8 May, 2024}
}.
This also makes Ireland one of the few countries in the world to suffer population decline over the past 170 years when the world's population has increased more than 6 fold
\footnote{
	\href{https://www.dfa.ie/irish-embassy/usa/about-us/ambassador/ambassadors-blog/black47irelandsgreatfamineanditsafter-effects/}
	{Blog by Ambassador Mulhall on Black'47: Ireland's Great Famine and its after-effects, Last accessed: 9 May, 2024}
}. Regarding the land, on the one hand, in the aftermath of the famine, there was a tendency in Ireland to shift from agriculture to livestock husbandry
\footnote{
	\href{https://www.cso.ie/en/statistics/othercsopublications/farmingsincethefamine1847-1996/}{CSO: Farming Since the Famine, 1847 - 1996, Last accessed: 12 May, 2024}
}, and on the other hand, when the late blight back in the 1870s, the Land War, which was directed at the landowners and the government, took place at the same time, with a deep consequences for the land structure of Ireland. And finally, there raised hostility between Irish and UK government, which was described as ``a bankruptcy of the British-Irish Union of 1800'' \citep{gray2021great}.

But data on Ireland's food imports and exports show increases in specific commodities, even barley, oats and butter, that violate the characteristics of the Great Famine. In History Ireland magazine, Christine wrote:
\begin{itemize}
	\item[] \textit{Almost 4,000 vessels carried food from Ireland to the ports of Bristol, Glasgow, Liverpool and London during 1847, when 400,000 Irish men, women and children died of starvation and related diseases [\ldots] The most shocking export figures concern butter [\ldots] That works out to be 822,681 gallons of butter exported to England from Ireland.}\footnote{
	\href{https://www.ighm.org/learn.html}{Ireland's Great Hunger Museum: Learn About the Great Hunger, Last accessed: 13 May, 2024}}
\end{itemize}
Scholars pondered if potato blight was the root cause of the famine, and they have engaged in many discussions about the origin factor, like Catholic and religious behavior \citep{miller1975irish}, anti-Irish racism \citep{waters1995great}, the poor law and colonial bio-politics \citep{nally2008coming} and, typically, the potato blight \citep{bartoletti2001black}, etc.

Although this day, it is certain that the root causes of the Great Famine, and the following decreasing of population were multiple regardless of the perspective used, historically, the academic attribution of the famine changed \citep{henderson2005irish}:

\vspace{0pt}
\begin{table}[h]
    \centering
	\begin{threeparttable}
    \caption{Timeline of the Great Famine and Population Decrease Attribution}
		\begin{tabular}{lp{5cm}p{5cm}}
			\toprule
			Timeline & Root Cause Summary & Reference \\
			\midrule
			1845 \textendash\ 1852: famine & Few food importation and opposition in poor law & 1850/01/05 The Illustrated London News\tnote{a} \\
			& & \\
			1852 \textendash\ 1920: neglected  & \textemdash\ \tnote{b} & \citep{kinealy2017great}\\
			& & \\
			1920 \textendash\ 1960: nationalist & Key grouping, like landlord class or the UK government & \citep{smith2005reckoning} \\
			& & \\
			1960 \textendash\ 1980: revisionism & Focus on history and event itself, ignore outside force & \citep{daly2006revisionism} \\
			& & \\
			1980s: post-revisionist & Emotional description also blame UK government & \citep{hamera2011outline} \\
			& & \\
			1980s: diverse & Malthus population theory & \citep{o2009food} \& \citep{mcgregor1989demographic} \& \citep{weir1991malthus} \\
			& &  \\
			& Anti-Malthus theory & \citep{o1983malthus} \& \citep{mokyr1980malthusian} \& \citep{guinnane1994great}\\
			& & \\
			& Blight biological analysis & \citep{donnelly2011irish}\\
			& & \\
			& Foucault's bio-politics and colonial perspective & \citep{nally2008coming} \& \citep{kennedy2020beckett} \& \citep{madden2016aids} \\
			\bottomrule
		\end{tabular}
		\begin{tablenotes}
			\item[a] The original newspaper mentioned: \textit{Free importation of corn into this union is essentially necessary \textendash\ [\ldots] any attempt to re-impose a duty on the importation of food can only [\ldots]  tend to the starving of the people. Poor law [\ldots] relieves the struggling farmer of a heavy burden he had hitherto.} \citep{1850_01_05_news}
			
			\vspace{7pt}

			\item[b] The famine literature few. The quantity and quality of work on the famine sparse: \textit{The two standard books of the Great Famine, [\ldots] the chapters were uneven in quality and lacked coherence (some lacked footnotes, some were lost).} \citep{kinealy2017great}

		\end{tablenotes}
	\end{threeparttable}
\end{table}

\vspace{-5pt}

The narrative travel along the path of Irish history. When nationalism was high, there was a tendency to external attribution; then when the economy and society stabilized, revisionism was born. As Hu Shih, a Chinese philosopher of the 1900s, put it, \textit{Reality, like a block of marble in our hands, is carved into whatever likeness we choose}.



What these strands of history described is that while food shortages are an objective fact, there are nonetheless other causes that conspire to drive famine \textendash\ as Amartya Sen's rights approach asserts.

%Based on the theoretical structure described above, this paper would like to reject some of the established theories on the famine \textbf{(Chapter 2.1)} and propose an Amartya Sen entitlement approach perspective on the Irish Famine \textbf{(Chapter 2.2)}. Then this paper will discuss the data used in this paper and its collection process \textbf{(Chapter 3)}, present the RDD regression methodology employed \textbf{(Chapter 4)} and then verify the applicability of the rights approach to this scenario \textbf{(Chapter 5)}. Finally, a conclusion will be presented \textbf{(Chapter 6)}.

