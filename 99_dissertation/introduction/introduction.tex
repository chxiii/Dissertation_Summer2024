\chapter{Introduction}
\textit{October playing a symphony on a slack wire paling.\\
Maguire watches the drills flattened out\\
And the flints that lit a candle for him on a June altar,\\
Flameless. \\
\textemdash\ ``The Great Hunger'' by Patrick Kavanagh.} \citep{Kavanagh_Quinn_2006a}
\vspace{.2cm}

The Irish Great Famine (1845 \textendash\ 1852) reshaped the entire history of Ireland. Before the Great Famine, according to the 1841 census, the population of the Ireland had close to 8.5 million
\footnote{
	\href{https://www.cso.ie/en/statistics/historicalreports/census1841/}{1841 Census of Ireland, Last accessed: 13 May, 2024}
}
. In 1851, when the Irish Great Famine had not yet ended, census noted that about 1 million people had died for hunger, and a similar number had gone into overseas exile
\footnote{
	\href{https://www.cso.ie/en/statistics/historicalreports/census1851/}{1851 Census of Ireland, Last accessed: 2 May, 2024}
}. In 1926, as a result of the Irish independence 5 years earlier, the Central Statistical Office was capable to integrate historical documents since famine and showed the fact that the population was decline of roughly 22\%
\footnote{
	\href{https://www.cso.ie/en/media/csoie/census/census1926results/volume10/C_1926_V10_Chapter_II.pdf}{1926 Census of Ireland, Chapter II, Last acceseed: 9 May, 2024}
} in the 10 years from 1841 to 1851. Using parish baptism data, some scholars have estimated that in the year 1847 alone \textendash\ which is also known as black 47 in Ireland history \textendash\ there existed counties with a nearly 70\% reduction in baptisms in Munster province in the south of Ireland \citep{cousens1960regional}, especially from southwest Cork and including north and east Clare
\footnote{
	\href{https://www.rte.ie/history/the-great-irish-famine/2020/0629/1150367-the-great-irish-famine/}{RTE, How "a truly modern famine" devastated Ireland, Last accessed: 11 May, 2024}
}
, while it was not the worst hit by the famine compared to the province of Connacht in the west
\footnote{
	\href{https://www.wesleyjohnston.com/users/ireland/past/famine/summer_1847.html}{Wesley Johnston: The Famine: The Summer of Black'47, Last accessed: 13 May, 2024}
}
. Apart from these quantitative explorations, the Great Famine is equally pivotal in Irish cultural history and ethnography. From Joseph O'Connor's fiction ``Star of the sea'' to W. B. Yeats's ``The Countess Cathleen'', together they expressed that the Great Famine not only pointed to the corpses of the dead, but also to a black hole of identity, naming and meaning \citep{luchen41naming}. 

The effects of the Great Famine were far-reaching, and reflected in the long-term population development, land institution structure and attitude to the UK government directly. It was not until 120 years later, in the 1960s, that Ireland's population began to grow consistently due to large-scale emigration, late marriage and a high incidence of permanent celibacy no longer hold \citep{grada1979population}, but it was still nowhere near as large as it had been during the Great Famine
\footnote{
	\href{https://www.cso.ie/en/releasesandpublications/ep/p-cpsr/censusofpopulation2022-summaryresults/populationchanges/}
	{2022 Census of Ireland \textendash\ Summary Results, Last accessed: 8 May, 2024}
}.
This also makes Ireland one of the few countries in the world to suffer population decline over the past 170 years when the world's population has increased more than six fold
\footnote{
	\href{https://www.dfa.ie/irish-embassy/usa/about-us/ambassador/ambassadors-blog/black47irelandsgreatfamineanditsafter-effects/}
	{Blog by Ambassador Mulhall on Black'47: Ireland's Great Famine and its after-effects, Last accessed: 9 May, 2024}
}. Regarding the 



People believe potato blight was responsible for the Irish Great Famine. 

lumper potato