\chapter{Introduction}
\textit{October playing a symphony on a slack wire paling.\\
Maguire watches the drills flattened out\\
And the flints that lit a candle for him on a June altar,\\
Flameless. \\
\textemdash\ ``The Great Hunger'' by Patrick Kavanagh.} \citep{Kavanagh_Quinn_2006a}
\vspace{.5cm}

The effects of the Great Famine were far-reaching. It wasn't until 120 years later, in the 1960s, that Ireland's population began to grow consistently due to the disappear of large-scale emigration, late marriage and a high incidence of permanent celibacy \citep{grada1979population}, but it was still nowhere near as large as it had been during the Great Famine
\footnote{
	\href{https://www.cso.ie/en/releasesandpublications/ep/p-cpsr/censusofpopulation2022-summaryresults/populationchanges/}
	{Census of Population 2022 \textendash\ Summary Results, Last accessed: 8 May, 2024}
}, 
and this also makes Ireland one of the few countries in the world to suffer population decline over the past 170 years when the world's population has increased more than six fold
\footnote{
	\href{https://www.dfa.ie/irish-embassy/usa/about-us/ambassador/ambassadors-blog/black47irelandsgreatfamineanditsafter-effects/}
	{Blog by Ambassador Mulhall on Black '47: Ireland's Great Famine and its after-effects, Last accessed: 9 May, 2024}
}.


